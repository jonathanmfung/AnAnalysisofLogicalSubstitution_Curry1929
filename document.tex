\documentclass[10pt, twoside]{extarticle}
% equivalent to textwidth=4.75in (need to explicitly define left/right to center twoside doc)
\usepackage[paperheight=9.38in, paperwidth=6.33in, left=0.79in, right=0.79in, bottom=1.2in]{geometry}

\usepackage{titlesec}
\usepackage[symbol, perpage]{footmisc}
\usepackage[inline]{enumitem}
\usepackage[leqno]{amsmath}
\usepackage{amsthm}
\usepackage{amssymb}
\usepackage{fancyhdr}
\usepackage{todonotes}
\usepackage{tocloft}
\newcommand{\td}[2][] {\todo[tickmarkheight=3pt, inline, size=\tiny, #1]{#2}}

%%%%%%%%%%%%%%%%%%%%%%%%%%%%%%%%%%%%%%%%
%%%%%%%%%%%%% TOC  %%%%%%%%%%%%%%%%%%%%
\setcounter{tocdepth}{0} % toc show only parts
\setlength{\cftbeforetoctitleskip}{0pt}

%% Center TOC title
\renewcommand{\cfttoctitlefont}{\hfill\normalfont\normalsize\textit}
\renewcommand{\cftaftertoctitle}{\hfill}

%% Part entry formatting
\setlength{\cftbeforepartskip}{0pt}
\setlength{\cftpartindent}{2em}

% https://tex.stackexchange.com/a/510274
\setlength{\cftpartnumwidth}{1.5em}
\renewcommand{\cftpartpresnum}{\hfill}% Right-align
\renewcommand{\cftpartaftersnum}[1]{}% Remove \hfill after number
\renewcommand{\cftpartaftersnumb}{\hspace{1em}}% Space between number and section title
\renewcommand{\cftpartfont}{\normalsize}

%% fake because need to insert only number in body
% https://tex.stackexchange.com/a/129985
\newcommand{\fakepart}[1]{%
  \par\refstepcounter{part}% Increase part counter
  \partmark{#1}% Add part mark (header)
  \phantomsection
  \addcontentsline{toc}{part}{\protect\numberline{\thepart.}#1}% Add part to ToC
  \begin{center}
    \large\bfseries \thepart.
  \end{center}
}

\makeatletter
\@addtoreset{section}{part}
\makeatother

\renewcommand{\thesection}{\Roman{section}.}
\titleformat*{\section}{\normalsize\scshape}

%%%%%%%%%%%%%  MATH  %%%%%%%%%%%%%%%%%%%
% description lists
\newcommand\litem[1]{\item{\textit{#1}}}

\newcommand\longeq{=\joinrel=}

\newtheorem{theorem}{Theorem}
\newtheorem{lemma}{Lemma}
\newtheorem{definition}{Definition}
\newtheorem{corollary}{Corollary}
% TODO: maybe remove QED symbol \renewcommand\qedsymbol{QED}
% TODO: \newtheoremstyle https://en.wikibooks.org/wiki/LaTeX/Theorems#Theorem_styles

%%%%%%%%%%%%%  MISC  %%%%%%%%%%%%%%%%%%%
\title{An Analysis of Logical Substitution}% Only for metadata
\author{Haskell Brooks Curry}% Only for metadata

% Make footnote mark not superscript https://tex.stackexchange.com/a/698131
\makeatletter
\renewcommand{\@makefnmark}{\hbox{\small\@thefnmark}}
\makeatother

%%%%%%%%%%%%%%%%%%%%%%%%%%%%
\begin{document}%

% custom footnoterule size https://tex.stackexchange.com/a/37884
\makeatletter
\renewcommand\footnoterule{%
  \kern-3\p@
  \hrule\@width6em
  \kern2.6\p@}
\makeatother

\begin{center}
\large\textbf{An Analysis of Logical Substitution.} \\
\normalsize By H. B. Curry. \\
\rule{5em}{0.7pt}
\end{center}

\renewcommand{\contentsname}{Contents.\footnote{The three parts of this paper are to a certain extent independent of one another. However, certain definitions needed in Part III are given in the last six paragraphs of Part II.} \hfill}
\tableofcontents

% NOTE: fancy pagestyle must be after tableofcontents
\pagestyle{fancy}
\renewcommand{\headrulewidth}{0em}
\fancyhead{} % clear all header fields
\fancyhead[C]{\textsc{Curry}: \textit{An Analysis of Logical Substitution.}}
\fancyhead[LE, RO]{\thepage}
\fancyfoot{} % clear all footer fields

\td{slightly more line height (<5\%)}
\td{slightly condensed font (<5\%)}

\renewcommand{\abstractname}{\vspace{-\baselineskip}}% https://tex.stackexchange.com/a/53175
\begin{abstract}
  \begin{itshape}
    \noindent Originally published in XXX YYY by HBCurry. According to U.S. laws, the original work is public domain as of Jan 1, 2025.
    This document is a digital recreation in \LaTeX{}, view the source repo here.
    Typeset by Jonathan Fung, contact at ZZZZZZZZZZ.
    The work was recreated as faithfully as possible, but is not pixel-perfect.
  \end{itshape}
\end{abstract}

\fakepart{Preliminary Discussion of the Nature of Mathematical Logic.}
\indent Mathematical Logic has been defined as an application of the formal
methods of mathematics to the domain of Logic.\footnote{Hilbert, D., and Ackermann, W., Grundziige der theoretischen Logik, 1928, p. 1}\td{proper citations in footnotes} Logic, on the other hand,
is the analysis and criticism of thought.\footnote{Johnson, W. E., Logic, Part I, Cambridge (1921), p. xiii} In accordance with these defini-
tions, the essential purpose of mathematical logic is the construction of an
abstract (or strictly formalized) theory, such that when its fundamental
notions are properly interpreted, there ensues an analysis of those universal
principles in accordance with which valid thinking goes on. The term
analysis here means that a certain rather complicated body of knowledge is
exhibited as deriveable from a much simpler body assumed at the beginning.
Evidently the simpler this initial knowledge, and the more explicitly and
carefully it is set forth, the more profound and satisfactory is the analysis
concerned.

In the present paper I propose to take some preliminary steps toward a
theory of logic in which the assumed initial knowledge is simpler than in
any existing theory with which I am acquainted. Before this is done, however,
it is necessary to consider somewhat in detail what is meant by the phrase
``abstract theory,'' and what is the significance of such a theory for the
analysis of thought. The object of this discussion is to see just how the
assumed knowledge enters into the theory; for this purpose we shall need to
be explicit, even at the risk of repeating what has already been better said
by others.

Certain ideas concerning the nature of an abstract theory can be disposed
of at once. In the first place the naive notion that such a theory consists of
a set of primitive ideas and propositions together with their consequences
by the laws of pure logic, must be dismissed on the ground of its circularity.
Again it is said that an abstract theory is one from which all meaning has
been abstracted. This requires that the sense of the term meaning be ex-
plained. If we take the term meaning, as applied to objects, to signify the
totality of properties (of those objects) which are directly apprehensible to
our intuition, then every object presented to the mind has meaning, and a
meaningless theory is a contradiction in terms. Even a symbol cannot be
meaningless in this sense; for either it denotes some object, or else it is
itself the object, and so has meaning. If we use the word meaning in some
other sense, then it loses its significance as related to the assumed initial
knowledge of our theory. Consequently the idea of a meaningless theory
must be subjected to further scrutiny.

Let us use the word meaning, as applied to concepts, in the sense of the
preceding paragraph. Then, relative to a given theory, we may distinguish
two kinds of meanings, which we shall call natural and conventional meanings
respectively. Natural meanings are those which are comprehensible a priori
in terms of our previous knowledge; conventional meanings those based on
relations to the theory itself. Natural meanings we may further subdivide
into essential and accidental: essential meanings are those on which the
deduction of the theory depends; accidental meanings those which are non-
essential. The distinction between these three kinds of meaning is important
in what follows.

The distinction between natural and conventional meanings has a
counterpart in that between statements of fact and statements of convention.
By a statement of fact I mean something of which truth or falsehood can
significantly be predicated; by a statement of convention a declaration of
intention, definition, or the like. The former corresponds to an act of judg-
ment,the latter to one of volition. Common sense and grammar have long
recognized both of these types; yet logicians seem to belittle the latter in that they define the proposition so as to exclude it.\footnote{See Johnson, W. E., 1. c., p. 1. Johnson's definition of the proposition is what I have given as the definition of a statement of fact.} Both these kinds of state-
ment, however, are equally intelligible to a rational mind; in this sense it is
false to say that one of them is less significant than the other. As examples
of statements of convention we have of course the definitions of technical
terms; but not all statements of convention are verbal---for instance the rules
of chess, which, by a sufficient amount of circumlocution, may be stated
without defining any new terms whatever. The postulates of any branch of
mathematics are of this character.

Let us now return to the abstract theory. I suggest that such a theory
is characterized by the following: \begin{enumerate*}[label=\arabic*)] \item the explicit indication of all essential
meanings; \item the absence, or at least omission from consideration, of
accidental meanings; \item the circumstance that the statements with which
the theory begins are conventional, and are, furthermore, sufficiently detailed
so that all the acts necessary to the deduction are specified.
\end{enumerate*}

To be yet more precise,an abstract theory begins with a set of primitive
notions, which, taken collectively, we shall call the \textit{primitive frame}, as follows:

\section[Non-Formal Primitive Ideas.]{Non-Formal Primitive Ideas.\footnote{The term idea is used here to denote an object, not a process of thought.}}
A set of ideas to each of which a certain amount of essential meaning is
attached, although they need not coincide with any ideas previously enter-
tained.\footnote{\textit{I. e.}, their meaning may be partly conventional.} For example:

\begin{enumerate}[label=\arabic*.,font=\itshape,wide]
\litem{Entities.} In order for an object to be considered in the theory at
all, it must have some property; this fact we may express by saying it is
an entity of one sort or another. These properties must then be among the
primitive ideas of the theory; and they must have essential meaning in
that they are predicates. In the simplified theory only one such notion is
necessary; but in the more complicated ones there are several; e. g. in the
Principia Mathematica there are individual, proposition, function, etc., the
latter two of various orders and types.

\litem{Modes of Combination.} I.e. processes by means of which entities may
be combined to get new entities. These have essential meaning in that they
are combinations. It must be specified by rules that the results of combina-
tion are entities. In the simple cases only one such notion is necessary, and
that a dyadic one; in the more complicated cases the various processes of
substitution are of this nature.

\litem{Assertions.} An assertion is a kind of entity, which is of special
importance because the object of deduction is to derive new assertions. The
idea of assertion has essential meaning only in that it is a predicate applicable
to certain entities. Ordinarily an assertion is interpreted as a statement to
which belief attaches, but this meaning is accidental.
\end{enumerate}

\section{Formal Primitive Ideas.}
Ideas which have no essential meanings (except that they are concepts).
They must of course be entities and their relations to other parts of the
primitive frame will give them conventional meanings.

\section{Postulates.}
All propositions of the theory are statements that certain particular
entities are assertions; the postulates are the propositions, if any, which are
assumed at the beginning. They are purely conventional.

\section{Rules.}
Statements of the processes by means of which new entities\footnote{Strictly speaking we should consider in the theory not only statements that an entity is an a-ssertion, but also statements that such and such combinations are entities. But the latter are, in simple cases at least, of so trivial a nature that it is not necessary to give them special prominence.} or new
propositions maybe constructed. Such statements are of course conventional;
moreover they are universal statements (involving the notion of ``every''
or its equivalent\footnote{Otherwise the rule would make possible the addition of only a finite number of constituents, and these could just as well be added explicitly to the preceding categories of the primitive frame}). They thus differ from propositions not only in that
they involve intuitive ideas from which the propositions are free, but also in
that they form the methods of transition, rather than the stopping places in
the theory. A typical example is the ``rule of inference'' which may be
stated thus: whenever \(p\) and \(p \supseteq q\) are assertions, then \(q\) shall also be an
assertion
\td{vertical gap}

In addition to the above notions there are yet to be considered those
associated with the use of symbolism. Whether these are to be regarded as
a part of the theory or as something superposed upon it, is a question which
I prefer to leave to the reader to adopt such views as seem best to him.
However he may decide, certainly language is necessary in order that the
theory may be communicated. The use of this language may involve intui-
tive operations other than those we have mentioned; it is desirable that these,
too, be specified by rules; because otherwise it is not certain that intuitive,
knowledge, other than that expressly mentioned, does not creep into the theory.
We shall call such rules \textit{symbolic conventions}.

So much for the primitive frame. The abstract theory itself may now be
defined as the doctrine built upon such a primitive frame by means of the
following processes: \begin{enumerate*}[label=\arabic*)] \item the derivation of new propositions, each of which is of
the form that such and such an entity is an assertion, by means of the rules;
\item the addition of new ideas by definitions. \end{enumerate*} The latter process may be
regarded either as a symbolic matter, governed by symbolic conventions, or as
the introduction of a new idea along with postulates and rules to the effect
that it is identical with some already existing entity. It is worth emphasiz-
ing that since statements that entities are not assertions do not occur among
the propositions, such a theory can never lead to a contradiction there.

The importance of such a theory for the analysis of thought lies in the
definiteness with which the intuitive knowledge entering into it is set forth.
Indeed, so far as the abstract theory itself is concerned, the only knowledge
assumed is the appreciation of the essential meanings and conventional state-
ments appearing in the primitive frame. When the theory is interpreted the
additional knowledge that must be brought to bear consists of the following:
that the concepts which we substitute for the primitive ideas have the neces-
sary essential meanings, and that the conventional statements in the primitive
frame correspond to facts. In both cases the required information is precisely
specified.

On the philosophic nature of such a theory, its relations to the symbolism
used in its expression, and to the various concrete theories obtained by inter-
preting it, it suffices to say that such questions are largely metaphysical, and
therefore irrelevant to the present discussion. It is by no means self-evident
that the best interests of science are served by adopting any one theory to the
exclusion of all others; any more than it is desirable that two persons follow-
ing the same argument should have the same mental imagery.\footnote{In writing the foregoing account I have naturally made use of any ideas I may have gleaned from reading the literature. The writings of Hilbert are fundamental in this connection. I hope that I have added clearness to certain points where the existing treatments are obscure.}

The next point to which I wish to direct the reader's attention is the car-
dinal importance of the rules in any abstract theory related to logic. For the
amount of initial knowledge which enters into the first threecategories of the
primitive frame is slight. In the rules, however, such knowledge is involved
in every step of the construction; for we have to pass judgment as to whether
the contemplated act is or is not according to Hoyle. These judgments, more-
over, are the only ones which are required. The rules, therefore, form the
port of entry of intelligence; and since nothing can be done without them
they represent the atoms of thought, so to speak, into which the reasoning
can be decomposed. It follows that in constructing such a theory it is not
sufficient merely to reduce the postulates and primitive ideas to their lowest
terms; it is even more important to so chose the rules that they involve,in
their application, only the simplest actions of the human mind.

Now although the rule of inference, stated above, is simple enough, yet in
all current mathematical logics there exist rules which are highly complex.
The presence of these complex rules raises the question whether it is possible
to formulate a theory which is---\begin{enumerate*}[label=\arabic*)] \item adequate for the whole of logic, \item based
on a finite number of primitive ideas, postulates, and rules, the last of the
same order of complexity as the rule of inference. \end{enumerate*} I believe that it is; indeed
steps in that direction have already been taken.\footnote{See the paper of Schönfinkel cited below.} As a preliminary to treating
this general problem, I shall discuss in the rest of this paper a special one con-
nected with it; viz., the analysis of the process of substitution. The latter
process is one of those complicated rules which occur in practically every
logical theory to-day.

The reader will observe that in the theory which results from the analysis
the formulas are more complicated, and the deductions required to produce
them more lengthy, than would be the case in the older theory. This is
inevitable. Indeed if we are to dissect the reasoning into microscopic pieces
it is but natural that more of them should be necessary to bring about a given
result. Consequently we must adopt a point of view suggested by Hilbert.
With each theory there is associated a metatheory in which we reason intui-
tively about the theory. In this metatheory we can derive more and more
complicated rules by showing, in general terms, how any particular conse-
quence of the derived rules can actually be deduced from the primitive ones.
The aim of mathematical logic is, in fact, not to reduce mathematics to a
formalism, à la \textit{Principia}, in which all steps explicitly appear; but rather to
analyze logic with a view to obtaining a greater command over its use, and a
more profound understanding of its nature. In this paper we shall adopt
this metatheoretic point of view.

\fakepart{Logical Substitution; its Relation to a Combinatory Problem.}

The process of substitution referred to above is the insertion of a constant
entity for one or more of the variables in a propositional function. The com-
plexity of this process is manifest. For not only is a function of \(n\) variables
a distinct concept for every value of \(n\), but the constant may be inserted in
any one of the \(n\) places, and each such insertion is a distinct act; further-
more, in connection with the universal and existential prefixes,we have a
process which virtually amounts to the simultaneous substitution of an entity
in two or more distinct places, as a procedure distinct from any of the fore-
going. The process of substitution is therefore compound in that it is not
one maneuver, but many. Moreover, when these acts of substitution are per-
formed in succession, there are many equivalences between the different possi-
bilities. To take the simplest example: suppose in a given function \(\phi(x, y)\)
we substitute \(a\) for \(x\), and in the result, which is \(\phi(a, y)\), we substitute \(b\) for \(y\),
 we have the proposition \(\phi(a, b)\); on the other hand, if we first substitute \(b\)
for \(y\) and then \(a\) for \(x\), we obtain the very same proposition. Yet these two
processes are in no sense identical. The substitution process is therefore not
only compound but complex, in the sense that it has structure. Thus there
is a considerable amount of information presupposed by the process; and
the rules involving it cannot have the maximum possible simplicity.

A notion closely related to substitution is that of transformation of func-
tions. Suppose we regard a function as having inherent in its definition a
certain order of its variables. Then permuting these variables in any way,
or making two or more of them alike, will produce new functions related to
the old; let us call them transforms of the original function, and the opera-
tions by which they are produced transformations. If we number the varia-
bles consecutively \(1, 2, 3, \cdots,\) then the transforms for a function of two
variables will be---
\[\phi(1, 2), \phi(2, 1), \phi(1, 1).\]

For three variables there will be 13 transforms, for four variables 75, for five
variables 541, etc. It is clear that the process of substituting a series of con-
stants in an arbitrary manner (such that the total number of entities counting
repetitions is \(n\)) into the original function is equivalent to the substitution of
the same entities in a prescribed manner (viz., the first entity into the place
of the first variable, the second into that of the second, etc.) into one of the
transforms. The study of substitution is thus to a certain degree equivalent
to the study of these transformations.

An important step toward the analysis of this situation was made by
M. Schönfinkel.\footnote{``Ueber die Bausteine der mathematischen Logik,'' Mathematische Annalen, Vol. 92 (1924), pp. 305-316.} Starting, apparently, from the fact that every logical for-
mula is a combination of constants---the variables being only apparent---he
shows that neither the notions of propositional function (of various orders)
nor that of substitution need be assumed as primitive; his formulation of
logic is such that variables, real or apparent, do not appear explicitly. His primitive frame is essentially as follows:\footnote{In this presentation I have changed Schönfinkel's formulation in some matters of detail.}

\section{Non-Formal Primitive Ideas.}
\begin{enumerate}[label=\arabic*.,font=\itshape,wide]
\litem{Entity}---not mentioned by Schönfinkel, but to be understood
essentially as a single notion of the sort mentioned in the general description
of an abstract theory above.

\litem{Application}---a mode of combination, the only one in the theory.
Two entities \(x\) and \(y\) combine to give a third entity called the application of
\(x\) to \(y\) and denoted by \((xy)\). The \textit{interpretation} of this is as follows: if \(x\) is a function,
then \((xy)\) is the result of substituting \(y\) for the first variable in \(x\);
thus if \(f\) denote a function of one variable, \((fx)\) denotes what is ordinarily
written \(f(x)\), if \(f\) is a function of two variables, \(((fx) y)\) denotes what is
ordinarily written \(f(x, y)\), etc. Nothing is said concerning the interpretation
of \((xy)\) when \(x\) is not a function; if the reader is disturbed over this lack,
he may invent one arbitrarily, \textit{e. g.} \((xy)\) may then be equal to \(x\).

\litem{Assertion.} To be understood as in the general description of an
abstract theory. Not denoted by any particular symbol; but when a symbol
of the form \((((\longeq)x)y)\) \((\text{or } x \longeq y)\) where \(x\) and \(y\) may be quite complicated,
stands out by itself like an equation in algebra, then the proposition that the
corresponding entity is an assertion is to be understood.
\end{enumerate}

\section{Formal Primitive Ideas.}
Three, denoted by \((\longeq)\), \(S\) and \(K\). In the interpretation \((\longeq)\) is to cor-
respond with identity; \(S\) and \(K\) are operations in the sense defined by the
rules.
\td{vertical gap}
\noindent \textit{Symbolic Conventions.}
\td{more spacing in math/parenthesis}
\begin{enumerate}[label=\arabic*.]
\item If \(x\) and \(y\) are any entities whatever, then instead of \((((\longeq)x)y)\)
we may write \((x\longeq y)\).
\item If \(x_1, x_2, \cdots, x_n\) are any entities, then instead of
\[((\cdots (((x_1 x_2)x_3)x_4)\cdots)x_n) \]
we may write \((x_1 x_2 x_3 \cdots, x_n)\).
\item The outside parentheses may be left off in the case of a symbol stand-
ing by itself or on either side of the sign \(\longeq\).
\end{enumerate}

\section{Postulates.}
None.
\section{Rules.}

\begin{enumerate}[label=\arabic*., start=0]
\item If \(x\) and \(y\) are entities, then \((xy)\) shall be an entity.
\item \((\longeq)\) shall have the properties of identity. These properties may be
specified by a few simple rules; but in this treatment we shall not go into
that detail. We shall treat \((\longeq)\) as if it were precisely the intuitive relation
of equality.
\item If \(x\) and \(y\) are any entities, then
  \[Kxy \longeq x\]
\item If \(x,y,z\) are entities, then
  \[Sxyz \longeq xz(yz)\]
\item If \(X\) and \(Y\) are combinations of \(S\) and \(K\), and if there exists an
  integer \(n\) such that by application of the preceding rules we can formally
  reduce the expressions \(Xx_1 x_2 \cdots x_n\) and \(Yx_1 x_2 \cdots x_n\) to combinations of
  \(x_1 x_2 \cdots x_n\) which have the same structure, then \(X \longeq Y\).
\end{enumerate}

If the above primitive frame were a part of a general theory of logic, the
term entity would include not only the various combinations of \(S\) and \(K\), but
all the notions of logic as well. In the sequel we shall accordingly speak of
the application of combinations \(S\) and \(K\) to various logical notions, and of the
resulting notions to each other, just as if these notions had been adjoined to
the above frame.

The \textit{raison d'être} of the theory based on this frame is the following fact:
Let \(x_1, x_2,\cdots, x_n\) be any \(n\) entities, and \(X\) any combination of them con-
structed by means of application. Then there exists a unique \(Y\), which is a
combination of \(S\) and \(K\) and independent of \(x_1, x_2,\cdots, x_n\) such that
\[X \longeq Y x_1 x_2 \cdots x_n.\]

When we recall the interpretation to be given to application we have the fol-
lowing result: given any logical formula built up from functions \(f_1, f_2,\cdots, f_m\)
and variables \(x_1, x_2,\cdots, x_n\) by substitution and rearrangement in any man-
ner; then the formula is expressible in the form
\[Fx_1x_2\cdots x_m\]
where
\[F\longeq Yf_1f_2\cdots f_m.\]
Now as already remarked, in the formulas expressing propositions of logic,
the variables are only apparent; which means that they are only a device by
means of which rather complicated relations among the logical constants
may be expressed; these relations, as the preceding argument shows, may also
be expressed by means of the operators Y, so that when the Schönfinkel theory
is used it is not necessary that variables should appear at all.

The theory of transformation of functions is included in the above as a
special case; viz., when there is a single function \(f\) and \(F\) is a transform.

The theory is, however, open to objection from our point of view because
of the complexity of Rule 4\td{link}. This rule is not mentioned by Schönfinkel; but
it is necessary in order that the \(Y\) mentioned be unique. In fact, the combina-
tions \(SK\) and \(K(SKK)\) determine the same \(X\); yet it is evidently not possi-
ble to establish their identity by means of the first four rules.

This situation suggests a problem; viz., to find a set of postulates which,
when adjoined to the Schönfinkel frame, enable us to dispense with Rule 4.
In what follows we shall obtain the solution of a special case under this prob-
lem; specifically, we shall find a set of postulates such that, within a certain
subclass of combinations of \(S\) and \(K\), all the \(Y\)'s which correspond
to the same \(X\) may be proved equal by means of these postulates and Rules 0-3. The sub-
class is one which has particular reference to logical substitution.

To begin with, we make the following definitions (the first three were made by Schönfinkel):\footnote{We use \(B\) and \(C\) respectively in place of Schönfinkel's \(Z\) and \(T\). Nothing cor- responding ot \(W\) or \(C_2, C_3, \cdots\) is defined by him.}

\begin{align*}
  I &\longeq SKK \\
  B &\longeq S(KS)K \\
  C_1 &\longeq S(BBS)(KK) \\
  C_2 &\longeq BC_1; C_3 \longeq BC_2; \cdots \text{etc.} \\
  W &\longeq SS(SK)
\end{align*}
then the reader may verify that whenever \(x_0, x_1, x_2 \cdots\) are entities
\begin{equation*}
  \renewcommand{\arraycolsep}{0pt}% Remove separation between array columns
  \begin{array}{rl}
  Ix_0 &\longeq x_0 \\
  Bx_0x_1x_2 &\longeq x_0(x_1x_2) \\
  C_1x_0x_1x_2 &\longeq x_0x_2x_1 \\
  C_2x_0x_1x_2x_3 &\longeq x_0x_1x_3x_2 \\
  \hdotsfor[5]{2} \\
  Wx_0x_1 &\longeq x_0x_1x_1
  \end{array}
\end{equation*}
Also we define multiplication, thus
\[X\cdot Y \longeq BXY\]
This multiplication is associative, and furthermore with respect to it, \(I\) is an
identical element These properties follow from Rule 4; they may also be
proved from the postulates:
\begin{align*}
  &C_1(BB(BBB))B \longeq BBBBB \\
  &BI \longeq I \\
  &C_1BI \longeq I. \\
\end{align*}

The subclass in question shall now consist of all the combinations formed by multiplication from the \(C\)'s, \(W\), and \(I\).

Before discussing the significance of this subclass, we turn aside to make
some conventions in regard to transformations. Let us redefine a transforma-
tion as an operation converting the sequence \(1, 2, 3, \cdots\) into the sequence
\(a_1, a_2, a_3, \cdots\) The latter sequence will, in the cases we consider, have the
property that there exists and index \(m\) such that:
\begin{enumerate*}[label=\arabic*)]
\item for \(i \leqq m\) the \(a_i\) are a
  permutation, with repetitions allowed, but not omissions, of the integers from
  \(1\) to \(m - p\), \(p \geqq 0\)\td{long minus sign}, inclusive, \item for \(i > m, a_i \longeq i - p\). \end{enumerate*} Let \(m\) be the least
integer having this property; then we shall call \(m\) the order of the trans-
formation. We shall denote a transformation by writing in brackets the
sequence into which it transforms the sequence \(1, 2, 3, \cdots\); furthermore,
there is no ambiguity if we indicate only the first \(n\) terms of the former
sequence, if \(m \leqq n\).

We shall further agree that a transformation of order \(m\) may operate on
a function of any number of variables \(\geqq m\); and that the effect of the trans-
formation on the function \(\phi(1, 2, \cdots, n)\), where \(n \geqq m\), shall be the trans-
form \(\phi(a_1, a_2, \cdots a_n)\). We shall regard as undefined the effect of a trans-
formation on a function of multiplicity less than the order of the transforma-
tion. Then there is one and only one transformation which carries a given
function into a given transform.

Multiplication of transformations we now define as follows:\td{indent equation numbers}
\begin{equation}
  \label{eq:multiplication_transformations}
[a_1, a_2, a_3 \cdots] \cdot [b_1, b_2, b_3 \cdots] \longeq [a_{b_1}, a_{b_2}, a_{b_3} \cdots]
\end{equation}

\td{no indent}(The product transformation has a finite order, provided that the factors do.)

Suppose, now, that we have a combination, \(X\), of \(S\) and \(K\), having the
property that there exists a transformation\td{check alpha boldness} \(\alpha \longeq [a_1, a_2 \cdots a_m]\) such that for
arbitrary \(x_0, x_1, x_2, \cdots x_{m-p}\)
\[Xx_0 x_1 x_2 \cdots x_{m-p} \longeq x_0 x_{a_1}x_{a_2} \cdots x_{a_m}\]
where \(m\) and \(p\) are defined as above. Then we shall say that \(X\) corresponds to
\(alpha\). It follows from Rules 0-3 that if \(X\) and \(Y\) are entities which correspond to
transformations \(\alpha\) and \(\beta\), respectively, then \(X \cdot Y\) corresponds to \(\alpha\beta\).

I now assert that every entity of our subclass corresponds to a transforma-
tion; and conversely that every \(X\) corresponding to a transformation, is an
entity of the subclass. For since the generating entities, the \(C\)'s, \(W\), and \(I\),
correspond to transformations, the first part of the statement follows from the
closing sentence of the last paragraph. The last part follows similarly from
the fact that every transformation can be obtained by multiplication from
permutations of adjacent integers and the transformation \([1, 1.]\) Of course
the one-to-oneness of the correspondence depends essentially on Rule 4.

The problem now is to find a set of postulates for the \(C\)'s and \(W\), from
which we may conclude, without the use of Rule 4, that the correspondence
between our subclass and the set of all transformations is a simple isomorph-
ism. To this we now turn.

\fakepart{Solution of the Combinatory Problem.}

\td{theorem: italicize + newline + no number}
\td{is textit redundant in theorems?}
\td{is equality sign for numbers same as terms?}
\begin{theorem}
  \begin{enumerate}[label=\arabic*)]
\item \textit{Let \(\mathfrak{A}\) be a system of operators, in which there exists an associative
    multiplication and an identical element, \(I\).}
\item \textit{Let this system be generated by operators \(W, C_1, C_2, \cdots\)} Subject to the postulates
  \td{more width in postulates}
  \begin{alignat*}{3}
    \text{I. }& C_i \cdot C_i \longeq I &&i \longeq 1, 2, 3, \cdots \\
    \text{II. }& C_i \cdot C_{i+1} \cdot C_i \longeq C_{i+1} \cdot C_i \cdot C_{i+1} &&i \longeq 1, 2, 3, \cdots \\
    \text{III. }& C_i \cdot C_j \longeq C_j \cdot C_i &&i \longeq 1, 2, 3, \cdots && j > i + 1 \\
    \text{IV. }& C_i \cdot W \longeq W \cdot C_{i+1} &&i \longeq 2, 3, 4, \cdots \\
    \text{V. }& W \cdot C_1 \longeq W \\
    \text{VI. }& W \cdot W \cdot C_2 \longeq W \cdot W\footnote{VI and VII are equivalent repectively to (6) and (7) (see below). The latter may if desired replace VI and VII.} \\
    \text{VII. }& W \cdot D_j \cdot W \longeq D_{j-1} \cdot W \cdot D_3 \cdot W \cdot C_2 \cdot C_3 \cdot C_1 \cdot C_2 \footnote{TODO: same} &&j \longeq 3, 4, 5 \cdots
  \end{alignat*}
  where
  \begin{align*}
    &D_1 \longeq I \\
    &D_{j+1} \longeq C_j \cdot D_j \longeq C_j \cdot C_{j-1} \cdot C_{j-2} \cdots \cdots C_1
  \end{align*}
  \td{multiple (5) cdots}

\item \textit{Let a correspondence be set up between the system \(\mathfrak{A}\) and the system of transformations of finite order as follows:}
  \begin{align}
    W &\sim [1, 1] \nonumber \\
    C_1 &\sim [2, 1] \label{eq:finite_transformations} \\
    C_2 &\sim [1, 3, 2] \nonumber \\
    C_i &\sim [1, 2, 3, \cdots i - 1, i + 1, i] \nonumber
  \end{align}
  and if\td{long sim (https://tex.stackexchange.com/a/19385)} \(A \sim \alpha\), \(B \sim \beta\)\footnote{Throughout this discussion we use Roman capitals to denote operators of \(\mathfrak{A}\), Greek l. c. letters to denote transformations.} then \(A \cdot B \sim \alpha \cdot \beta\) is defined by \eqref{eq:multiplication_transformations}.
  \end{enumerate}
  Then, the correspondence so defined is a one-to-one isomorphism.
\end{theorem}
\begin{proof}
  \td{newline after proof}
  There are four things to prove:
  \begin{enumerate}[label=\arabic*)]
  \item That to every product expression in \(\mathfrak{A}\) there corresponds a unique
    transformation.
  \item That this correspondence is an isomorphism.
  \item That if two transformations correspond to two expressions in \(\mathfrak{A}\) which
    are equal by Virtue of I-VII (inclusive), then these transformations are equal.
  \item That conversely there corresponds an operator of \(\mathfrak{A}\) to each transfor-
    mation, and any two operators corresponding to the same transformation may
    be proved equal by I-VII.
  \end{enumerate}

  Of these four things the first three follow immediately: the first two
  by definitions; the third because propositions analogous to I-VII are true
  \setcounter{footnote}{1} % NOTE: for some reason this page of 3 footnotes runs into "Counter Too Large"
  % https://tex.stackexchange.com/a/342621
  for transformations.\footnote{The only question here is about VII. It follows, however, by definition, that
    \[D_i \sim [i, 1, 2, 3, \cdots, i-1]\]
    and therefore the proposition analogous to VII is the identity
    \begin{align*}
      [1, 1] \cdot [j, 1, &2, \cdots, j-1] \cdot [1,1] \\
      &\longeq [j-1, 1, 2, \cdots, j-2] \cdot [1. 1] \cdot [3, 1, 2] \cdot [1, 1] \cdot [3, 4, 1, 2].
    \end{align*}
    Both sides here are equal to \([j-1, j-1, 1, 1, 2, 3, \cdots, j-2].\)
  }

  \noindent It remains, therefore, only to prove the fourth.

  By a theorem due to E. H. Moore\footnote{Proceedings of the London Mathematical Society, Vol. 28 (1897), p. 357-366. Moore's proof involves some rather complicated theorems in group theory. It is, however, possible to prove the theorem by directly establishing the correspondence.} postulates I-III insure that the
  subset of \(\mathfrak{A}\) which is obtained from the first \(n-1\) \(C\)'s only is isomorphic
  with the symmetric group on n letters; the correspondence is established in
the same way as here. It follows readily that the subset generated by all the
\(C\)'s is isomorphic with the totality if all those transformations which do
not allow repetition. In fact, suppose two such operators of \(\mathfrak{A}\) correspond
to the same transformation; let \(n\) be the largest index appertaining to a \(C\)
in either one; then the two operators correspond to the same permutation on
\(n + 1\) letters, and therefore, by Moore's result, are equal in virtue of I-III.
In the following we shall assume this result as known; and therefore shall
often designate operators composed of \(C\)'s only by means of the transforma-
tions corresponding to them.

In order to prove the fourth statement above when \(W\) is considered, we
shall show that every operator in \(\mathfrak{A}\) can be expressed in a certain normal
form, and then that one and only one such expression in normal form
corresponds to a given transformation.

To obtain such a normal form we introduce the following definitions:
\td{dotted hrule}
\begin{align*}
  W_1 &\longeq W \\
  W_2 &\longeq C_1 \cdot W_1 \cdot C_2 \cdot C_1 \\
  W_{k+1} &\longeq C_k \cdot W_k \cdot C_{k+1} \cdot C_k \\
\end{align*}
it then follows by induction\footnote{This and other proofs by mathematical deduction are such that by repeating the argument a sufficient number of timmes the formula may be established in any particular case with the use of I-VII.} that
\begin{align*}
  W_k &\longeq D_k \cdot W \cdot C_2 \cdot C_3 \cdot C_4 \cdots C_k \cdot C_1 \cdot C_2 \cdot C_3 \cdots C_{k-1} \\
  &\longeq [k, 1, 2, \cdots k-1] \cdot W \cdot [3, 4, \cdots k+1, 1, 2]
\end{align*}

whence,\td{dotted hrule, long sim}
\begin{align*}
  W_2 &\sim [1, 2, 2] \\
  W_3 &\sim [1, 2, 3, 3] \\
  W_k &\sim [1, 2, 3, \cdots k-1, k, k] \\
\end{align*}

The result we wish to prove now follows from a series of lemmas.
\begin{lemma}
  Every product expression in \(\mathfrak{A}\) can be reduced to the form
  \begin{equation}
    \label{eq:product_reduction}
    W_{k_q} \cdot W_{k_{q-1}} \cdots W_{k_2} \cdot W_{k_1} \cdot B.
  \end{equation}
\end{lemma}
\begin{proof}
First we show that any expression of the form \(A \cdot W\) can be reduced to the
form \(W_k \cdot B\), where \(A\) and \(B\) involve the \(C\)'s only. If \(A\) is the identical ele-
ment we are through. If not, consider the \(C\) nearest the \(W\); if this is any
other than \(C_1\) it can be passed across the \(W\) by IV, otherwise we have an
expression of the form \(A' \cdot C \cdot W\), where \(A'\) involves one less \(C\) than \(A\).

Now suppose w have reduced \(A \cdot W\) to the form \(A' \cdot D_j \cdot W \cdot B '\) where
\(A'\) and \(B'\) involve the \(C\)'s only and \(A'\) is not \(I\). Let \(C_i\) be that \(C\) in \(A'\)
which is nearest \(D_j\). Four cases can rise: \begin{enumerate*}[label=\arabic*)] \item if \(i < j-1\), \(C_i \cdot D_j \longeq D_j \cdot C_{i+1}\)
  and since \(i + 1 > 1\), \(C_{i+1}\) can be passed across the \(W\) by IV;
\item if \(i \longeq j-1\), \(C_i \cdot D_j \longeq D_{j-1}\); \item if \(i \longeq j\), \(C_i \cdot D_j \longeq D_{j+1}\); \item if \(i > j\),
  \(C_i \cdot D_j \longeq D_j \cdot C_i\), and since \(i > j \geqq 1\), \(C_i\) can be passed across the \(W\) by
  IV.\end{enumerate*} In all four cases we have reduced the expression to another one of the
  same form where the new \(A'\) involves one less \(C\) than the old.

  Continuing in this way we must eventually reach a stage where \(A' \longeq I\). Then we have

  \td{maybe make flalign for ``by I/def''}
  \begin{alignat*}{2}
    A \cdot W &\longeq D_k \cdot W \cdot B' \\
              &\longeq (D_k \cdot W \cdot C_2 \cdot C_3 \cdots C_k \cdot C_1 \cdot C_2 \cdots C_{k-1}) \\
              &\qquad \qquad \qquad \cdot C_{k-1} \cdots C_2 \cdot C_1 \cdot C_k \cdots C_3 \cdot C_2 \cdot B' \text{ by I}\\
              &\longeq W_k \cdot C_{k-1} \cdots C_2 \cdot C_1 \cdot C_k \cdots C_3 \cdot C_2 \cdot B' \text{ by def.}
  \end{alignat*}
  which is of the form \(W_k \cdot B\).

  The proof of the lemma now follows. Let the given expression be of the form
  \begin{equation}
    \label{eq:4}
A_q \cdot W \cdot A_{q-1} \cdot W \cdots A_2 \cdot W \cdot A_1 \cdot W \cdot A_0
\end{equation}
where \(A_0, A_1, A_2 \cdots A_q\) involve the \(C\)'s only. By what we have just proved,

\begin{equation*}
  \renewcommand{\arraycolsep}{0pt}% Remove separation between array columns
  \begin{array}{rl}
    A_q \cdot W &{}\longeq  W_{k_q} \cdot B_q \cdot \\
    B_q \cdot A_{q-1} \cdot W &{}\longeq  W_{k_{q-1}} \cdot B_{q-1} \\
    B_{q-1} \cdot A_{q-2} \cdot W &{}\longeq  W_{k_{q-2}} \cdot B_{q-2} \\
    \hdotsfor[11]{2} \\
    B_2 \cdot A_1 \cdot W &{}\longeq  W_{k_1} \cdot B_1 \\
    B_1 \cdot A_0 &{}\longeq  B.
  \end{array}
\end{equation*}

Taking all these into consideration, we can reduce the form \eqref{eq:4} to the form \eqref{eq:product_reduction}.
\end{proof}

\begin{lemma}
  Every expression in \(\mathfrak{A}\) can be reduced further to a form \eqref{eq:product_reduction}, in which\td{multiple dots}
  \[k_q \geqq k_{q-1} \geqq \cdots \cdots \geqq k_2 \geqq k_1\]
\end{lemma}

\begin{proof}
  It is sufficient to prove that
  \begin{equation}
    \label{eq:5}
    W_j \cdot W_{k+1} \longeq W_k \cdot W_j \qquad \text{ for } k \geqq j.
  \end{equation}
  The proof of \eqref{eq:5} is as follows.
\end{proof}

\noindent \textit{First,}
\begin{flalign}
  \label{eq:6}
    && W_1 \cdot W_2 &\longeq W_1 \cdot W_1. &\\
    &\text{For} \quad & W_1 \cdot W_2 &\longeq W_1 \cdot C_1 \cdot W_1 \cdot C_2 \cdot C_1 &\text{by def.} \nonumber \\
                                    &&&\longeq W_1 \cdot W_1 &\text{by V and VI.} \nonumber
\end{flalign}

\noindent \textit{Second,}
\begin{flalign}
  \label{eq:7}
    && W_1 \cdot W_{k+1} &\longeq W_k \cdot W_1 \quad\text{ for } k > 1.&&\\
    &\text{For} &W_1 \cdot W_{k+1} &\longeq W_1 \cdot D_{k+1} \cdot W_1 \cdot [3, 4, \cdots k+2, 1, 2]  &\text{by def.} \nonumber \\
    &&&\longeq D_k \cdot W_1 \cdot [3, 1, 2] \cdot W \cdot [3,4,1,2] & \nonumber\\
    &&&\qquad \qquad\cdot [3,4,\cdots k + 2, 1, 2] &\text{by VII.}
  \nonumber
\end{flalign}

\begin{flalign*}
  &\text{But since}  &&[3, 4, 1, 2] \cdot [3, 4, \cdots k+2, 1, 2] &\\
                    &&& \qquad \qquad \longeq [1, 2, 5, 6, \cdots k+2, 3, 4] &\\
  &\text{and}        &&W_1 \cdot [1, 2, 5, 6, \cdots k+2, 3, 4] &\\
                    &&& \qquad \qquad \longeq [1, 4, 5, \cdots k+1, 2, 3] \cdot W_1 & \text{by IV.}\\
                    &&&[3, 1, 2] \cdot [1, 4, 5, \cdots k+1, 2, 3] &\\
                    &&& \qquad \qquad \longeq [3, 4, \cdots k+1, 1, 2] &\\
  \intertext{we have}
                    &&&W_1 \cdot W_{k+1} \longeq D_k \cdot W_1 \cdot [3, 4, \cdots k+1, 1, 2] \cdot W &\\
                    &&& \qquad \qquad \longeq W_k \cdot W_1 &\text{q. e. d.}
\end{flalign*}

\noindent \textit{Third,}
\begin{equation}
  \label{eq:8}
  C_k \cdot W_j \longeq W_j \cdot C_k \qquad \text{ for } k < j-1.
\end{equation}
\indent For by definition
\[C_k \cdot W_j \longeq C_k \cdot D_j \cdot W \cdot [3, 4, \cdots j+1, 1, 2].\]

But we have
\begin{align*}
  &C_k \cdot D_j \longeq D_j \cdot C_{k+1} \\
  &C_{k+1} \cdot W \longeq W \cdot C_{k+2} \\
  &C_{k+2} \cdot [3, 4, \cdots j+1, 1, 2] \longeq [3, 4, \cdots j+1, 1, 2] \cdot C_k
\end{align*}

\noindent (the first and third relations follow from the known properties of permut-
tions, the second from IV.) Combining these three we have \eqref{eq:8}.

\noindent \textit{Fourth,}
\begin{equation}
  \label{eq:9}
  W_j \cdot C_j \longeq W_j.
\end{equation}
For \(j \longeq 1\), this is true by V.

Suppose, now, \eqref{eq:9} is true for \(j-1\) (i. e. suppose \(W_{j-1}\cdot C_{j-1} \longeq W_{j-1}\)) then
\begin{flalign*}
  && W_j \cdot C_j &\longeq C_{j-1} \cdot W_{j-1} \cdot C_j \cdot C_{j-1} \cdot C_j & \text{by def.} \\
                 &&&\longeq C_{j-1} \cdot W_{j-1} \cdot C_{j-1} \cdot C_j \cdot C_{j-1} & \text{by II.} \\
                 &&&\longeq C_{j-1} \cdot W_{j-1} \cdot C_j \cdot C_{j-1} & \text{by hyp.} \\
                 &&&\longeq W_{j-1}& \text{by def.}
\end{flalign*}

\noindent \textit{Fifth,}
\begin{equation}
  \label{eq:10}
  C_{j+h} \cdot W_j \longeq W_j \cdot C_{j+h+1} \qquad \text{for } h>0
\end{equation}
For \(j\longeq 1\), this is true by IV. Suppose it true for \(j-1\), then
\begin{flalign*}
  && C_{j+h} \cdot W_j &\longeq C_{j+h} \cdot C_{j-1} \cdot W_{j-1} \cdot C_j \cdot C_{j-1} & \text{by def.} \\
                     &&&\longeq C_{j-1} \cdot C_{j+h} \cdot W_{j-1} \cdot C_j \cdot C_{j-1} & \text{by III.} \\
                     &&&\longeq C_{j-1} \cdot W_{j-1} \cdot C_{j+h+1} \cdot C_j \cdot C_{j-1} & \text{by hyp.} \\
                     &&&\longeq C_{j-1} \cdot W_{j-1} \cdot C_j \cdot C_{j-1} \cdot C_{j+h+1} & \text{by III.} \\
                     &&&\longeq W_j \cdot C_{j+h+1} & \text{q.e.d. \qquad by def.}
\end{flalign*}

\noindent \textit{Sixth,}
For \(j\longeq 1\), this is true by \eqref{eq:6}. Suppose it true for \(j-1\), then
\td{can't fit q.e.d on last line}
\td{goes past right margin}
\begin{flalign*}
  && W_j \cdot W_j &\longeq C_{j-1} \cdot W_{j-1} \cdot C_j \cdot W_{j-1} \cdot C_j \cdot C_{j-1} & \text{by def. and I.} \\
                 &&&\longeq C_{j-1} \cdot W_{j-1} \cdot W_{j-1} \cdot C_{j+1} \cdot C_j \cdot C_{j-1} & \text{by \eqref{eq:10}} \\
                 &&&\longeq C_{j-1} \cdot W_{j-1} \cdot W_j \cdot C_{j+1} \cdot C_j \cdot C_{j-1} & \text{by hyp.} \\
                 &&&\longeq C_{j-1} \cdot W_{j-1} \cdot C_j \cdot C_{j-1} \cdot C_{j-1} \cdot C_j \cdot W_j \cdot C_{j+1} \cdot C_j \cdot C_{j-1} & \text{by I.} \\
                 &&&\longeq W_j \cdot C_{j-1} \cdot W_{j+1} \cdot C_{j-1} & \text{by def.} \\
                 &&&\longeq W_j \cdot W_{j+1} &\text{by \eqref{eq:8} and I.}
\end{flalign*}

\begin{definition}
  \(W_k^r \longeq W_k \cdot W_k \cdots W_k \quad r \text{ times.}\)
\end{definition}
\begin{corollary}
  Every expression in \(\mathfrak{A}\) can be reduced to the form
  \begin{equation}
    \label{eq:11}
    W_{k_q}^{r_q} \cdot W_{k_{q-1}}^{r_{q-1}} \cdots W_{k_2}^{r_2} \cdot W_{k_1}^{r_1} \cdot B \\
  \end{equation}
  where \(\quad k_q > k_{q-1} > \cdots \cdots > k_2 > k_1, \qquad r_1 > 0 \)
\end{corollary}
\begin{proof}
  Simply collect the \(W\)'s with equal indices in \eqref{eq:product_reduction}.
\end{proof}

\begin{lemma}
  If two expressions of the form \eqref{eq:11} correspond to the same
  transformation, then they both have the same constants \(q, k_1, k_2, \cdots k_q,
  r_1 r_2, \cdots r_q.\)
\end{lemma}
\begin{proof}
  Let \[\Delta_q \longeq W_{k_q}^{r_q} \cdot W_{k_{q-1}}^{r_{q-1}} \cdots W_{k_2}^{r_2} \cdot W_{k_1}^{r_1}\]

  also let \[\Delta_q \sim \alpha \longeq [a_1, a_2, a_3 \cdots \cdots]\]

  Then for the explicit determination of the \(a_x\) we have
  \begin{alignat*}{5}
    \text{If\quad} && 0 &< x < k_1           && \text{\quad then \quad} && a_x&&\longeq x \\
             && k_1 &\leqq x \leqq k_1 + r_1 &&                         &&a_x &&\longeq k_1 \\
           && k_1 + r_1 &< x < k_2 + r_1     &&                         &&a_x&&\longeq x-r_1 \\
           &&           &                    &&\text{\enskip etc.}      &&   &&
  \end{alignat*}
  In general, if we define
  \begin{align*}
    s_i &\longeq r_1 + r_2 + \cdots + r_1 \\
    r_0 &\longeq 0, s_0 \longeq 0, k_0 \longeq 0
  \end{align*}
  then, for \(i \longeq 1, 2, \cdots q\).
  \td{originally aligned on first k}
  \begin{alignat*}{5}
    \text{If\quad} && k_{i-1}+s_{i-1} &< x < k_i +s_{i-1}           && \text{\quad then \quad} && a_x&&\longeq x-s_{i-1} \\
             && k_i +s_{i-1} &\leqq x \leqq k_i + s_i &&                         &&a_x &&\longeq k_i \\
           && k_q +s_q &< x  &&                         &&a_x&&\longeq x-s_q.
  \end{alignat*}
  These formulas follow directly from \eqref{eq:multiplication_transformations} and \eqref{eq:finite_transformations}; their proofs are left to
  the reader.

  The transformation has thus the following character: \begin{enumerate*}[label=\arabic*)] \item the integers in the symbol \([a_1, a_2, a_3 \cdots]\) are arranged in their natural order with certain
    repetitions, \item the only integers which appear more than once are \(k_1, k_2 \cdots k_q\),
    and these appear respectively \(r_1 +1, r_2 +1, \cdots r_q +1\) times.
  \end{enumerate*}

  Now if  \(\beta \longeq [b_1, b_2 \cdots]\) corresponds to \(B\), then the transformation
  corresponding to \(\Delta_q \cdot B\) is \(\alpha \cdot \beta \longeq [a_{b_1}, a_{b_2}, \cdots]\). By the restrictions on \(B\)
  the sequence \([a_{b_1}, a_{b_2}, \cdots]\) is merely a rearrangement of the sequence
  \([a_1, a_2 \cdots]\). Hence the second of the above properties applies just as much
  to \(\alpha \cdot \beta\) as to \(\alpha\). But then the constants \(q, k_1 \cdots k_q, r_1 \cdots r_q,\) are uniquely
  determined by \(\alpha \cdot \beta\), \mbox{q. e. d.}
\end{proof}

\begin{lemma}
  To each transformation of finite order there corresponds at
  least one expression of form \eqref{eq:11}.
\end{lemma}
\begin{proof}
  Let the given transformation be \(\gamma \longeq [c_1, c_2 \cdots]\).

  Let \(\Delta_q\) be determined from \(\gamma\) as indicated in the preceding lemma, and
  suppose \(\Delta_q \sim [a_1, a_2 \cdots] \longeq \alpha\).

  Then we can construct a permutation \(\beta \longeq [b_1, b_2 \cdots]\) such that
  \(\alpha \cdot \beta \longeq \gamma\), as follows. If \(c_i\) is distinct from any of the \(k\)'s there is one and
  only one \(j\) such that \(a_j \longeq c_i\); in that case we must let \(b_i \longeq j\). On the other
  hand let the \(C\)'s which are equal to \(k_j\) be \(c_{i_1}, c_{i_2} \cdots c_{i_p}\) (where \(p \longeq r_j + 1\));
  then set \(b_{i_1} \longeq k_j + s_{j-1}, b_{i_2} \longeq b_{i_1} +1, b_{i_3} \longeq b_{i_2} +1,\) etc. For definiteness
  we may suppose that \(i_1 < i_2 < \cdots < i_p\). Then \(\alpha \cdot \beta \longeq \gamma\); for the i'th in-
  teger in the symbol for \(\alpha \cdot \beta\) is \(a_{b_i}\), and \(b_i\) has been so chosen that in all
  cases \(a_{b_i} \longeq c_i\).

  To the \(\beta\) so constructed there corresponds a unique \(B\) in \(\mathfrak{A}\), by Moore's
  result. For this \(B\), \(\Delta_q \cdot B \sim \gamma, q.e.d.\)
\end{proof}

\begin{lemma}
  The operator in \(\mathfrak{A}\) corresponding to a given transformation
  is unique.
\end{lemma}
\begin{proof}
  Let \(A_1\) and \(A_2\) be two operators in \(\mathfrak{A}\) which correspond to a given trans-
  \td{label/ref lemmas}
  formation \(\gamma\). By Lemmas 2 and 3 there is a uniquely determined \(\Delta_q\)
  such that
  \[A_1 \longeq \Delta_q \cdot B_1, \qquad A_2 \longeq \Delta_q \cdot B_2\]

  If \(B_1 \sim \beta_1, B_2 \sim \beta_2, \beta_1\) and \(\beta_2\) must be subject to all the restrictions
  to which \(\beta\) was subject in Lemma 4, except that it is not necessary to suppose
  \(i_1 < i_2 < \cdots < i_p\). The only way in which \(\beta_1\) and \(\beta_2\) can differ is in the
  arrangement of the \(b_{i_1}, b_{i_2} \cdots\) corresponding to each of the \(k_j\). Hence
  \[\beta_2 \longeq \beta_3 \cdot \beta_1\]
  where \(\beta_3\) is a product of permutations, each of which permutes among them-
  selves the integers composing one of the sets \(k_j + s_{j-1}, k_j + s_{j-1} + 1, \cdots k_j + s_j\).

  Let us now agree to denote by \(E_i^j, i<j\) a combinations of \(C_i, C_{i+1} \cdots C_{j-1}\)
  (corresponding to a permeation of \(i, i+1, \cdots j\)). Then, in view of the
  isomorphism already established by Moore, we have the following result;
  there exists a \(B_3\) such that
  \begin{align*}
    B_2 &\longeq B_3 \cdot B_1 \\
    B_3 &\longeq E_{k_1}^{k_1 + r_1} \cdot E_{k_2 + s_1}^{k_2 + s_2} \cdot E_{k_3 + s_2}^{k_3 + s_3} \cdots E_{k_q + s_{q-1}}^{k_q + s_q}
  \end{align*}
  \indent To prove the lemma it is sufficient to show that \(\Delta_q \cdot B_3 \longeq \Delta_q\). This, in
  turn, follows from the above form for \(B_3\), if we demonstrate that
  \begin{equation}
    \label{eq:12}
    \Delta_h \cdot E_{k_h + s_{h-1}}^{k_h + s_h} \longeq \Delta_k \qquad (h\longeq 1, 2, \cdots, q).
  \end{equation}

  \noindent We turn to this last question forthwith.

  In the first place, if
  \[ i > k+1, W_k \cdot E_i^j \longeq E_{i-1}^{j-1} \cdot W_k. \]
  This follows from \eqref{eq:10}, under Lemma 2.

  Next, if \(i>k+1\)
  \begin{equation}
    \label{eq:13}
    W_k^r \cdot E_i^j \longeq E_{i-r}^{j-r} \cdot W_k^r.
  \end{equation}
  This is derived from the preceding by induction on \(r\).

  Third, if \(i < h\)
  \begin{equation}
    \label{eq:14}
    \Delta_i \cdot E_{k_h + s_{h-1}}^{k_h + s_{h}} \longeq E_{k_h + s_{h-1} - s_i}^{k_h + s_h - s_i} \cdot \Delta_i.
  \end{equation}
  For \(i\longeq 1\), this follows from the preceding, for the conditions \(k_h + s_{h-1} > k_1 + r_1\)
  are satisfied since, if \(h > 1\), \(k_h > k_1\), \(s_{h-1} \geqq s_1 \longeq r_1\). For \(i > 1\)
  we prove \eqref{eq:14} by induction. Suppose it true for \(i-1\), then,
  \begin{align*}
    \Delta_i \cdot E_{k_h + s_h}^{k_h + s_{h-1}} &\longeq W_{k_i}^{r_i} \cdot \Delta_{i-1} \cdot E_{k_h + s_{h-1}}^{k_h + s_h} \\
                                                 &\longeq W_{k_i}^{r_i} \cdot E_{k_h + s_{h-1} - s_{i-1}}^{k_h + s_h - s_{i-1}} \cdot \Delta_{i-1}.
  \end{align*}
  In order, now, to apply \eqref{eq:13}, we need to know simply that
  \[k_h + s_{h-1} - s_{i-1} > k_i + r_i\]
  and this is fulfilled since for \(h > i\),
  \[s_{h-1} - s_{i-1} \longeq r_i + r_{i+1} + \cdots + r_{h-1} \geqq r_i.\]
  Hence
  \[W_{k_i}^{r_i} \cdot E_{k_h + s_{h-1} - s_{i-1}}^{k_h + s_h - s_{i-1}} \cdot \Delta_{i-1} \longeq E_{k_h + s_{h-1} - s_i}^{k_h + s_h - s_i} \cdot W_{k_i}^{r_i} \cdot \Delta_{i-1}\]
  so that \eqref{eq:14} is proved.

  The following is the special case of \eqref{eq:14} where \(i \longeq h-1\)
  \[\Delta_{h-1} \cdot E_{k_h + s_{h-1}}^{k_h + s_h} \longeq E_{k_h}^{k_h + r_h} \cdot \Delta_{h-1}.\]

  In order to complete the proof of \eqref{eq:12}, it remains (since \(\Delta_h \longeq W_{k_h}^{r_h} \cdot \Delta_{h-1}\))
  simply to show

  \begin{equation}
    \label{eq:15}
    W_{k_h}^{r_h} \cdot E_{k_h}^{k_h + r_h} \longeq W_{k_h}^{r_h}
  \end{equation}
  In view of the composition of \(E_{k_h}^{k_h + r_h}\), \eqref{eq:15} follows from
  \[W_k^r \cdot C_{k+j} \longeq W_k^r \qquad \text{for } 0 \leqq j< r.\]
  For \(j\longeq 0\), this is true by \eqref{eq:9}, under Lemma 2.
  For \(j\longeq 1\)
  \begin{flalign*}
    && W_k \cdot W_k \cdot C_{k+1} &\longeq W_k \cdot W_{k+1} \cdot C_{k+1} &\text{by \eqref{eq:5}} \\
                                &&&\longeq W_k \cdot W_{k+1}               &\text{by \eqref{eq:9}} \\
                                &&&\longeq W_k \cdot W_k                  &\text{by \eqref{eq:5}} \\
    &&\therefore \quad W_k^r \cdot C_{k+1} &\longeq W_k^r \qquad \qquad \text{for } r\geqq 2. &
  \end{flalign*}
  For \(j > 1\), let \(r \longeq s+j-1\), then \(s \geqq 2\) and
  \begin{flalign*}
    && W_k^r \cdot C_{k+j} &\longeq W_k^s \cdot W_k^{j-1} \cdot C_{k+j} &\\
                        &&&\longeq W_k^s \cdot C_{k+1} \cdot W_k^{j-1} &\text{by \eqref{eq:10}} \\
                        &&&\longeq W_k^s \cdot W_k^{j-1} &\text{by the previous case} \\
                        &&&\longeq W_k^r. &
  \end{flalign*}
  These three cases together establish \eqref{eq:15}, which completes the proof of both
  the lemma and the theorem.
\end{proof}% lemma
\end{proof}% theorem
We have now achieved one of our objectives---viz. the elimination of
Rule 4 so far as the selected subclass is concerned. For Rule 4 merely
enables us to conclude that two entities of the subclass, whose application to
a series \(x_0 x_1 x_2 \cdots\) yields the same transformation, are equal. These en-
tities correspond to the same transformation in the sense of our theorem,
and therefore by that theorem are equal. Moreover the proof of the theorem
is such that in any particular case a proof of the equality can be constructed
in terms of I-VII, together with the rules for identity and no more. This
proof will in general be quite long, but it may nevertheless be found.\footnote{MISSING FOOTNOTE IN ORIGINAL PAPER}

Finally we have an analysis of substitution in so far as that process is
equivalent to transformation.

There remain open two questions: \begin{enumerate*}[label=\arabic*)] \item What is the relation between the
  transforms of a function of \(n - 1\) variables, obtained by substituting a con-
  stant in a function of \(n\) variables, and the transforms of the original;
  \item How can our infinite set of postulates be reduced to a finite set. \end{enumerate*} Both
these questions can be answered by introducing the entity \(B\) defined in Part II.\td{link to part II}
But when we do that, it is expedient to consider a more extensive subclass.
This takes us out of the domain of simple substitution. The topic is left for
a later paper.
\end{document}

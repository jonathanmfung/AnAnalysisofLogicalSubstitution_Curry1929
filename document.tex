\documentclass[9pt]{article}
\usepackage[paperheight=9.38in, paperwidth=6.33in, textwidth=4.75in]{geometry}

\usepackage{titlesec}
\usepackage[symbol, perpage]{footmisc}
\usepackage[inline]{enumitem}
\usepackage{todonotes}
\newcommand{\td}[2][] {\todo[tickmarkheight=3pt, inline, size=\tiny, #1]{#2}}

%%%%%%%%%%%%%%%%%%%%%%%%%%%%%%%%%%%%%

% https://tex.stackexchange.com/a/129985
\newcommand{\fakepart}[1]{%
  \par\refstepcounter{part}% Increase part counter
  \partmark{#1}% Add part mark (header)
  \addcontentsline{toc}{part}{\protect\numberline{\thepart.}#1}% Add section to ToC
  \begin{center}
    \large\bfseries \thepart.
  \end{center}
}

\title{An Analysis of Logical Substitution}% Only for metadata
\author{Haskell Brooks Curry}% Only for metadata

%%%%%%%%%%%%%%%%%%%%%%%%%%%%
\begin{document}%

\listoftodos
\begin{center}
\large\textbf{An Analysis of Logical Substitution.} \\
\normalsize By H. B. Curry.
\end{center}

\renewcommand{\contentsname}{\hfill\normalfont\normalsize\textit{Contents}.\footnote{The three parts of this paper are to a certain extent independent of one another. However, certain definitions needed in Part III are given in the last six paragraphs of Part II.} \hfill}
\tableofcontents
\td{header (even left, same title)}
\td{shrink toc formatting}
\td{slightly more line height (<5\%)}
\td{slightly condensed font (<5\%)}


\fakepart{Preliminary Discussion of the Nature of Mathematical Logic.}\td{section title size}
\td{footnote mark size}
\td{indent} \indent Mathematical Logic has been defined as an application of the formal
methods of mathematics to the domain of Logic.\footnote{Hilbert, D., and Ackermann, W., Grundziige der theoretischen Logik, 1928, p. 1}\td{proper citations in footnotes} Logic, on the other hand,
is the analysis and criticism of thought.\footnote{Johnson, W. E., Logic, Part I, Cambridge (1921), p. xiii} In accordance with these defini-
tions, the essential purpose of mathematical logic is the construction of an
abstract (or strictly formalized) theory,such that when its fundamental
notions are properly interpreted, there ensues an analysis of those universal
principles in accordance with which valid thinking goes on. The term
analysis here means that a certain rather complicated body of knowledge is
exhibited as deriveable from a much simpler body assumed at the beginning.
Evidently the simpler this initial knowledge, and the more explicitly and
carefully it is set forth, the more profound and satisfactory is the analysis
concerned.
\td{footnote rule size}

In the present paper I propose to take some preliminary steps toward a
theory of logic in which the assumed initial knowledge is simpler than in
any existing theory with which I am acquainted. Before this is done, however,
it is necessary to consider somewhat in detail what is meant by the phrase
``abstract theory,'' and what is the significance of such a theory for the
analysis of thought. The object of this discussion is to see just how the
assumed knowledge enters into the theory; for this purpose we shall need to
be explicit, even at the risk of repeating what has already been better said
by others.

Certain ideas concerning the nature of an abstract theory can be disposed
of at once. In the first place the naive notion that such a theory consists of
a set of primitive ideas and propositions together with their consequences
by the laws of pure logic, must be dismissed on the ground of its circularity.
Again it is said that an abstract theory is one from which all meaning has
been abstracted. This requires that the sense of the term meaning be ex-
plained. If we take the term meaning, as applied to objects, to signify the
totality of properties (of those objects) which are directly apprehensible to
our intuition, then every object presented to the mind has meaning, and a
meaningless theory is a contradiction in terms. Even a symbol cannot be
meaningless in this sense; for either it denotes some object, or else it is
itself the object, and so has meaning. If we use the word meaning in some
other sense, then it loses its significance as related to the assumed initial
knowledge of our theory. Consequently the idea of a meaningless theory
must be subjected to further scrutiny.

Let us use the word meaning,as applied to concepts, in the sense of the
preceding paragraph. Then, relative to a given theory, we may distinguish
two kinds of meanings, which we shall call natural and conventional meanings
respectively. Natural meanings are those which are comprehensible a priori
in terms of our previous knowledge; conventional meanings those based on
relations to the theory itself. Natural meanings we may further subdivide
into essential and accidental: essential meanings are those on which the
deduction of the theory depends; accidental meanings those which are non-
essential. The distinction between these three kinds of meaning is important
in what follows.

The distinction between natural and conventional meanings has a
counterpart in that between statements of fact and statements of convention.
By a statement of fact I mean something of which truth or falsehood can
significantly be predicated; by a statement of convention a declaration of
intention, definition, or the like. The former corresponds to an act of judg-
ment,the latter to one of volition. Common sense and grammar have long
recognized both of these types; yet logicians seem to belittle the latter in that they define the proposition so as to exclude it.\footnote{See Johnson, W. E., 1. c., p. 1. Johnson's definition of the proposition is what I have given as the definition of a statement of fact.} Both these kinds of state-
ment, however, are equally intelligible to a rational mind; in this sense it is
false to say that one of them is less significant than the other. As examples
of statements of convention we have of course the definitions of technical
terms; but not all statements of convention are verbal---for instance the rules
of chess, which, by a sufficient amount of circumlocution, may be stated
without defining any new terms whatever. The postulates of any branch of
mathematics are of this character.

Let us now return to the abstract theory. I suggest that such a theory
is characterized by the following: \begin{enumerate*}[label=\arabic*)] \item the explicit indication of all essential
meanings; \item the absence, or at least omission from consideration, of
accidental meanings; \item the circumstance that the statements with which
the theory begins are conventional, and are, furthermore, sufficiently detailed
so that all the acts necessary to the deduction are specified.
\end{enumerate*}

To be yet more precise,an abstract theory begins with a set of primitive
notions, which, taken collectively, we shall call the \textit{primitive frame}, as follows:

\td{probably could be enumerations}
\td{Section in small caps}
\section[Non-Formal Primitive Ideas.]{Non-Formal Primitive Ideas.\footnote{The term idea is used here to denote an object, not a process of thought.}}
A set of ideas to each of which a certain amount of essential meaning is
attached, although they need not coincide with any ideas previously enter-
tained.\footnote{\textit{I. e.}, their meaning may be partly conventional} For example:

\td{proper counting in named items}
\begin{enumerate}[label=\arabic*., wide]
\item[\textit{1. Entities.}] In order for an object to be considered in the theory at
all, it must have some property; this fact we may express by saying it is
an entity of one sort or another. These properties must then be among the
primitive ideas of the theory; and they must have essential meaning in
that they are predicates. In the simplified theory only one such notion is
necessary; but in the more complicated ones there are several; e. g. in the
Principia Mathematica there are individual, proposition, function, etc., the
latter two of various orders and types.

\item[\textit{2. Modes of Combination.}] I.e. processes by means of which entities may
be combined to get new entities.These have essential meaning in that they
are combinations. It must be specified by rules that the results of combina-
tion are entities. In the simple cases only one such notion is necessary, and
that a dyadic one; in the more complicated cases the various processes of
substitution are of this nature.

\item[\textit{3. Assertions.}] An assertion is a kind of entity, which is of special
importance because the object of deduction is to derive new assertions. The
idea of assertion has essential meaning only in that it is a predicate applicable
to certain entities. Ordinarily an assertion is interpreted as a statement to
which belief attaches, but this meaning is accidental.
\end{enumerate}

\section{Formal Primitive Ideas.}
Ideas which have no essential meanings (except that they are concepts).
They must of course be entities and their relations to other parts of the
primitive frame will give them conventional meanings.

\section{Postulates.}
\section{Rules.}

\fakepart{Logical Substitution; its Relation to a Combinatory Problem.}
\[\phi(1, 2), \phi(2, 1), \phi(1, 1).\]
\fakepart{Solution of the Combinatory Problem.}

\end{document}

\documentclass[9pt]{article}
\usepackage[paperheight=9.38in, paperwidth=6.33in, textwidth=4.75in]{geometry}

\usepackage{titlesec}
\usepackage[symbol, perpage]{footmisc}
\usepackage[inline]{enumitem}
\usepackage{todonotes}
\newcommand{\td}[2][] {\todo[tickmarkheight=3pt, inline, size=\tiny, #1]{#2}}

%%%%%%%%%%%%%%%%%%%%%%%%%%%%%%%%%%%%%

% https://tex.stackexchange.com/a/129985
\newcommand{\fakepart}[1]{%
  \par\refstepcounter{part}% Increase part counter
  \partmark{#1}% Add part mark (header)
  \addcontentsline{toc}{part}{\protect\numberline{\thepart.}#1}% Add section to ToC
  \begin{center}
    \large\bfseries \thepart.
  \end{center}
}

\title{An Analysis of Logical Substitution}% Only for metadata
\author{Haskell Brooks Curry}% Only for metadata

%%%%%%%%%%%%%%%%%%%%%%%%%%%%
\begin{document}%

\listoftodos
\begin{center}
\large\textbf{An Analysis of Logical Substitution.} \\
\normalsize By H. B. Curry.
\end{center}

\renewcommand{\contentsname}{\hfill\normalfont\normalsize\textit{Contents}.\footnote{The three parts of this paper are to a certain extent independent of one another. However, certain definitions needed in Part III are given in the last six paragraphs of Part II.} \hfill}
\tableofcontents
\td{header (even left, same title)}
\td{shrink toc formatting}
\td{slightly more line height (<5\%)}
\td{slightly condensed font (<5\%)}


\fakepart{Preliminary Discussion of the Nature of Mathematical Logic.}\td{section title size}
\td{footnote mark size}
\td{indent} \indent Mathematical Logic has been defined as an application of the formal
methods of mathematics to the domain of Logic.\footnote{Hilbert, D., and Ackermann, W., Grundziige der theoretischen Logik, 1928, p. 1}\td{proper citations in footnotes} Logic, on the other hand,
is the analysis and criticism of thought.\footnote{Johnson, W. E., Logic, Part I, Cambridge (1921), p. xiii} In accordance with these defini-
tions, the essential purpose of mathematical logic is the construction of an
abstract (or strictly formalized) theory,such that when its fundamental
notions are properly interpreted, there ensues an analysis of those universal
principles in accordance with which valid thinking goes on. The term
analysis here means that a certain rather complicated body of knowledge is
exhibited as deriveable from a much simpler body assumed at the beginning.
Evidently the simpler this initial knowledge, and the more explicitly and
carefully it is set forth, the more profound and satisfactory is the analysis
concerned.
\td{footnote rule size}

In the present paper I propose to take some preliminary steps toward a
theory of logic in which the assumed initial knowledge is simpler than in
any existing theory with which I am acquainted. Before this is done, however,
it is necessary to consider somewhat in detail what is meant by the phrase
``abstract theory,'' and what is the significance of such a theory for the
analysis of thought. The object of this discussion is to see just how the
assumed knowledge enters into the theory; for this purpose we shall need to
be explicit, even at the risk of repeating what has already been better said
by others.

Certain ideas concerning the nature of an abstract theory can be disposed
of at once. In the first place the naive notion that such a theory consists of
a set of primitive ideas and propositions together with their consequences
by the laws of pure logic, must be dismissed on the ground of its circularity.
Again it is said that an abstract theory is one from which all meaning has
been abstracted. This requires that the sense of the term meaning be ex-
plained. If we take the term meaning, as applied to objects, to signify the
totality of properties (of those objects) which are directly apprehensible to
our intuition, then every object presented to the mind has meaning, and a
meaningless theory is a contradiction in terms. Even a symbol cannot be
meaningless in this sense; for either it denotes some object, or else it is
itself the object, and so has meaning. If we use the word meaning in some
other sense, then it loses its significance as related to the assumed initial
knowledge of our theory. Consequently the idea of a meaningless theory
must be subjected to further scrutiny.

Let us use the word meaning,as applied to concepts, in the sense of the
preceding paragraph. Then, relative to a given theory, we may distinguish
two kinds of meanings, which we shall call natural and conventional meanings
respectively. Natural meanings are those which are comprehensible a priori
in terms of our previous knowledge; conventional meanings those based on
relations to the theory itself. Natural meanings we may further subdivide
into essential and accidental: essential meanings are those on which the
deduction of the theory depends; accidental meanings those which are non-
essential. The distinction between these three kinds of meaning is important
in what follows.

The distinction between natural and conventional meanings has a
counterpart in that between statements of fact and statements of convention.
By a statement of fact I mean something of which truth or falsehood can
significantly be predicated; by a statement of convention a declaration of
intention, definition, or the like. The former corresponds to an act of judg-
ment,the latter to one of volition. Common sense and grammar have long
recognized both of these types; yet logicians seem to belittle the latter in that they define the proposition so as to exclude it.\footnote{See Johnson, W. E., 1. c., p. 1. Johnson's definition of the proposition is what I have given as the definition of a statement of fact.} Both these kinds of state-
ment, however, are equally intelligible to a rational mind; in this sense it is
false to say that one of them is less significant than the other. As examples
of statements of convention we have of course the definitions of technical
terms; but not all statements of convention are verbal---for instance the rules
of chess, which, by a sufficient amount of circumlocution, may be stated
without defining any new terms whatever. The postulates of any branch of
mathematics are of this character.

Let us now return to the abstract theory. I suggest that such a theory
is characterized by the following: \begin{enumerate*}[label=\arabic*)] \item the explicit indication of all essential
meanings; \item the absence, or at least omission from consideration, of
accidental meanings; \item the circumstance that the statements with which
the theory begins are conventional, and are, furthermore, sufficiently detailed
so that all the acts necessary to the deduction are specified.
\end{enumerate*}

To be yet more precise,an abstract theory begins with a set of primitive
notions, which, taken collectively, we shall call the \textit{primitive frame}, as follows:

\td{probably could be enumerations}
\td{Section in small caps}
\section[Non-Formal Primitive Ideas.]{Non-Formal Primitive Ideas.\footnote{The term idea is used here to denote an object, not a process of thought.}}
A set of ideas to each of which a certain amount of essential meaning is
attached, although they need not coincide with any ideas previously enter-
tained.\footnote{\textit{I. e.}, their meaning may be partly conventional} For example:

\td{proper counting in named items}
\begin{enumerate}[label=\arabic*., wide]
\item[\textit{1. Entities.}] In order for an object to be considered in the theory at
all, it must have some property; this fact we may express by saying it is
an entity of one sort or another. These properties must then be among the
primitive ideas of the theory; and they must have essential meaning in
that they are predicates. In the simplified theory only one such notion is
necessary; but in the more complicated ones there are several; e. g. in the
Principia Mathematica there are individual, proposition, function, etc., the
latter two of various orders and types.

\item[\textit{2. Modes of Combination.}] I.e. processes by means of which entities may
be combined to get new entities.These have essential meaning in that they
are combinations. It must be specified by rules that the results of combina-
tion are entities. In the simple cases only one such notion is necessary, and
that a dyadic one; in the more complicated cases the various processes of
substitution are of this nature.

\item[\textit{3. Assertions.}] An assertion is a kind of entity, which is of special
importance because the object of deduction is to derive new assertions. The
idea of assertion has essential meaning only in that it is a predicate applicable
to certain entities. Ordinarily an assertion is interpreted as a statement to
which belief attaches, but this meaning is accidental.
\end{enumerate}

\section{Formal Primitive Ideas.}
Ideas which have no essential meanings (except that they are concepts).
They must of course be entities and their relations to other parts of the
primitive frame will give them conventional meanings.

\section{Postulates.}
All propositions of the theory are statements that certain particular
entities are assertions; the postulates are the propositions, if any, which are
assumed at the beginning. They are purely conventional.

\section{Rules.}
Statements of the processes by means of which new entities\footnote{Strictly speaking we should consider in the theory not only statements that an entity is an a-ssertion, but also statements that such and such combinations are entities. But the latter are, in simple cases at least, of so trivial a nature that it is not necessary to give them special prominence.} or new
propositions maybe constructed. Such statements are of course conventional;
moreover they are universal statements (involving the notion of ``every''
or its equivalent\footnote{Otherwise the rule would make possible the addition of only a finite number of constituents, and these could just as well be added explicitly to the preceding categories of the primitive frame}). They thus differ from propositions not only in that
they involve intuitive ideas from which the propositions are free, but also in
that they form the methods of transition, rather than the stopping places in
the theory. A typical example is the ``rule of inference'' which may be
stated thus: whenever \(p\) and \(p \supseteq q\) are assertions, then \(q\) shall also be an
assertion
\td{vertical gap}

In addition to the above notions there are yet to be considered those
associated with the use of symbolism. Whether these are to be regarded as
a part of the theory or as something superposed upon it, is a question which
I prefer to leave to the reader to adopt such views as seem best to him.
However he may decide, certainly language is necessary in order that the
theory may be communicated. The use of this language may involve intui-
tive operations other than those we have mentioned; it is desirable that these,
too, be specified by rules; because otherwise it is not certain that intuitive,
knowledge, other than that expressly mentioned, does not creep into the theory.
We shall call such rules \textit{symbolic conventions}.

So much for the primitive frame. The abstract theory itself may now be
defined as the doctrine built upon such a primitive frame by means of the
following processes: \begin{enumerate*}[label=\arabic*)] \item the derivation of new propositions, each of which is of
the form that such and such an entity is an assertion, by means of the rules;
\item the addition of new ideas by definitions. \end{enumerate*} The latter process may be
regarded either as a symbolic matter, governed by symbolic conventions, or as
the introduction of a new idea along with postulates and rules to the effect
that it is identical with some already existing entity. It is worth emphasiz-
ing that since statements that entities are not assertions do not occur among
the propositions, such a theory can never lead to a contradiction there.

The importance of such a theory for the analysis of thought lies in the
definiteness with which the intuitive knowledge entering into it is set forth.
Indeed, so far as the abstract theory itself is concerned, the only knowledge
assumed is the appreciation of the essential meanings and conventional state-
ments appearing in the primitive frame. When the theory is interpreted the
additional knowledge that must be brought to bear consists of the following:
that the concepts which we substitute for the primitive ideas have the neces-
sary essential meanings, and that the conventional statements in the primitive
frame correspond to facts. In both cases the required information is precisely
specified.

On the philosophic nature of such a theory, its relations to the symbolism
used in its expression, and to the various concrete theories obtained by inter-
preting it, it suffices to say that such questions are largely metaphysical, and
therefore irrelevant to the present discussion. It is by no means self-evident
that the best interests of science are served by adopting any one theory to the
exclusion of all others; any more than it is desirable that two persons follow-
ing the same argument should have the same mental imagery.\footnote{In writing the foregoing account I have naturally made use of any ideas I may have gleaned from reading the literature. The writings of Hilbert are fundamental in this connection. I hope that I have added clearness to certain points where the existing treatments are obscure.}

The next point to which I wish to direct the reader's attention is the car-
dinal importance of the rules in any abstract theory related to logic. For the
amount of initial knowledge which enters into the first threecategories of the
primitive frame is slight. In the rules, however, such knowledge is involved
in every step of the construction; for we have to pass judgment as to whether
the contemplated act is or is not according to Hoyle. These judgments, more-
over, are the only ones which are required. The rules, therefore, form the
port of entry of intelligence; and since nothing can be done without them
they represent the atoms of thought, so to speak, into which the reasoning
can be decomposed. It follows that in constructing such a theory it is not
sufficient merely to reduce the postulates and primitive ideas to their lowest
terms; it is even more important to so chose the rules that they involve,in
their application, only the simplest actions of the human mind.

Now although the rule of inference, stated above, is simple enough, yet in
all current mathematical logics there exist rules which are highly complex.
The presence of these complex rules raises the question whether it is possible
to formulate a theory which is---\begin{enumerate*}[label=\arabic*)] \item adequate for the whole of logic, \item based
on a finite number of primitive ideas, postulates, and rules, the last of the
same order of complexity as the rule of inference. \end{enumerate*} I believe that it is; indeed
steps in that direction have already been taken.\footnote{See the paper of Schönfinkel cited below.} As a preliminary to treating
this general problem, I shall discuss in the rest of this paper a special one con-
nected with it; viz., the analysis of the process of substitution. The latter
process is one of those complicated rules which occur in practically every
logical theory to-day.

The reader will observe that in the theory which results from the analysis
the formulas are more complicated, and the deductions required to produce
them more lengthy, than would be the case in the older theory. This is
inevitable. Indeed if we are to dissect the reasoning into microscopic pieces
it is but natural that more of them should be necessary to bring about a given
result. Consequently we must adopt a point of view suggested by Hilbert.
With each theory there is associated a metatheory in which we reason intui-
tively about the theory. In this metatheory we can derive more and more
complicated rules by showing, in general terms, how any particular conse-
quence of the derived rules can actually be deduced from the primitive ones.
The aim of mathematical logic is, in fact, not to reduce mathematics to a
formalism, à la \textit{Principia}, in which all steps explicitly appear; but rather to
analyze logic with a view to obtaining a greater command over its use, and a
more profound understanding of its nature. In this paper we shall adopt
this metatheoretic point of view.

\fakepart{Logical Substitution; its Relation to a Combinatory Problem.}
\[\phi(1, 2), \phi(2, 1), \phi(1, 1).\]
\fakepart{Solution of the Combinatory Problem.}

\end{document}
